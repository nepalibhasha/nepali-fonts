\documentclass{article}
\usepackage[a4paper,margin=1in]{geometry}
\usepackage{nepali}

% Primary Nepali sans setup (repo-local files)
\setnepalifont[
  Path=../../fonts/sources/noto-sans-devanagari/,
  Extension=.ttf,
  UprightFont=NotoSansDevanagari-Regular,
  BoldFont=NotoSansDevanagari-Bold
]{NotoSansDevanagari-Regular}

% Additional Nepali families for mixed-font examples
\newfontfamily\nepaliserif[
  Path=../../fonts/sources/noto-serif-devanagari/,
  Extension=.ttf,
  UprightFont=NotoSerifDevanagari-Regular,
  BoldFont=NotoSerifDevanagari-Bold,
  ItalicFont=NotoSerifDevanagari-Light,
  Script=Devanagari,
  Language=Nepali
]{NotoSerifDevanagari-Regular}

\newfontfamily\nepalimukta[
  Path=../../fonts/sources/mukta/,
  Extension=.ttf,
  UprightFont=Mukta-Regular,
  BoldFont=Mukta-Bold,
  ItalicFont=Mukta-Light,
  Script=Devanagari,
  Language=Nepali
]{Mukta-Regular}

\begin{document}
\section*{Advanced Usage}
\textbf{Fonts used in this document:}\\
Noto Sans Devanagari (primary), Noto Serif Devanagari, Mukta.

\subsection*{1. Mixed Script Paragraph}
\textit{Active font in this section: Noto Sans Devanagari}\\
This line is English documentation text, but contains inline Nepali:
\textnepali{यो लाइन नेपालीमा पनि देखिन्छ।}
Build output should preserve script shaping and baseline rhythm.

\begin{nepali}
Version 2.0 मा converter logic सुधारिएको छ, तर release process अझै incremental छ।
JSON metadata, WOFF2 packaging, र CLI commands एउटै workflow मा प्रयोग भएका छन्।
\end{nepali}

\subsection*{2. Weight and Emphasis}
\textit{Active font in this section: Noto Sans Devanagari}\\
\begin{nepali}
सामान्य पाठ: यो नियमित (Regular) शैलीको उदाहरण हो।\\
\textbf{गाढा पाठ: यो Bold शैलीको उदाहरण हो।}\\
\textit{टेढो/Slanted पाठ: Devanagari मा true italic प्रायः हुँदैन, यो font-driven fallback हो।}\\
\textbf{\textit{गाढा + टेढो: संयुक्त emphasis उदाहरण।}}
\end{nepali}

\subsection*{3. Mixed Nepali Fonts in One Document}
\textit{Each line below explicitly switches to a different font family.}
{\nepalifont\textnepali{Noto Sans Devanagari: सफा र UI-friendly शैली।}}\\[0.4em]
{\nepaliserif\textnepali{Noto Serif Devanagari: पुस्तक-जस्तो serif बनावट।}}\\[0.4em]
{\nepalimukta\textnepali{Mukta: खुला आकार र आधुनिक दृश्य लय।}}

\subsection*{4. Mixed Fonts + Mixed Scripts}
\textit{Active Nepali font in this section: Noto Serif Devanagari}\\
{\nepaliserif
\textnepali{काठमाण्डूमा release pipeline सफल भयो;}}
CI status is \textbf{green}, and packaging target is
\texttt{@nepalibhasha/fonts}.

\subsection*{5. Nepali Numerals and Conjunct Stress}
\textit{Active font in this section: Noto Sans Devanagari}\\
\begin{nepali}
अंक: ० १ २ ३ ४ ५ ६ ७ ८ ९\\
संयुक्ताक्षर: क्ष ज्ञ त्र श्र द्ध ट्ट ण्ड स्त्र न्त्र
\end{nepali}

\subsection*{6. Sanskrit Verse Block (Subhashita)}
\textit{Active font in this section: Noto Serif Devanagari}\\
{\nepaliserif
\begin{nepali}
अयं निजः परो वेति गणना लघुचेतसाम्।\\
उदारचरितानां तु वसुधैव कुटुम्बकम्॥
\end{nepali}
}

\subsection*{7. Nepali Stanza for Flow Testing}
\textit{Active font in this section: Mukta}\\
{\nepalimukta
\begin{nepali}
पहाडको हावामा गुन्जिन्छ गाउँको पुरानो भाका,\\
बिहानीको घामसँगै जाग्छ श्रम र सपनाको कथा।\\
माटोको गन्धमा मिसिएको स्मृति र संघर्षको धुन,\\
शब्दहरूमा बाँच्छ हाम्रो भाषा, हाम्रो पहिचान, हाम्रो मन।
\end{nepali}
}

\end{document}
