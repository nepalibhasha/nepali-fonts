\documentclass{article}
\usepackage{nepali}

\begin{document}
\section*{Font Specimen}
\textbf{Fonts shown: Noto Sans Devanagari, Noto Serif Devanagari, Mukta}

\subsection*{Noto Sans Devanagari}
\textit{Font used in this section: Noto Sans Devanagari}
\setnepalifont[
  Path=../../fonts/sources/noto-sans-devanagari/,
  Extension=.ttf,
  UprightFont=NotoSansDevanagari-Regular,
  BoldFont=NotoSansDevanagari-Bold
]{NotoSansDevanagari-Regular}
\begin{nepali}
नेपाल नेपाली भाषा देवनागरी लिपिमा लेखिन्छ।
क्ष, ज्ञ, त्र, श्र, द्ध, ट्ट, ण्ड
० १ २ ३ ४ ५ ६ ७ ८ ९
\end{nepali}

\subsection*{Noto Serif Devanagari}
\textit{Font used in this section: Noto Serif Devanagari}
\setnepalifont[
  Path=../../fonts/sources/noto-serif-devanagari/,
  Extension=.ttf,
  UprightFont=NotoSerifDevanagari-Regular,
  BoldFont=NotoSerifDevanagari-Bold
]{NotoSerifDevanagari-Regular}
\begin{nepali}
नेपाल नेपाली भाषा देवनागरी लिपिमा लेखिन्छ।
क्ष, ज्ञ, त्र, श्र, द्ध, ट्ट, ण्ड
० १ २ ३ ४ ५ ६ ७ ८ ९
\end{nepali}

\subsection*{Mukta}
\textit{Font used in this section: Mukta}
\setnepalifont[
  Path=../../fonts/sources/mukta/,
  Extension=.ttf,
  UprightFont=Mukta-Regular,
  BoldFont=Mukta-Bold
]{Mukta-Regular}
\begin{nepali}
नेपाल नेपाली भाषा देवनागरी लिपिमा लेखिन्छ।
क्ष, ज्ञ, त्र, श्र, द्ध, ट्ट, ण्ड
० १ २ ३ ४ ५ ६ ७ ८ ९
\end{nepali}

\end{document}
